\documentclass[]{article}
\usepackage{fullpage}
\usepackage{amsmath}
\usepackage{graphicx}
\usepackage{listings}
\usepackage{hyperref}

\title{FOEuler User Guide}
\author{Mianzhi Wang}

\begin{document}

\maketitle

\tableofcontents

\section{Installing FOEuler}

\subsection{System Requirement}
\label{sec:requirement}

FOEuler is a component of the project FOSolverS, which is build upon other open-source packages.
The following tools and packages are required to build FOSolverS:
\begin{itemize}
  \item CMake, the build system used in project FOSolverS
  \item gcc with gfortran, the C and Fortran compiler used in project FOSolverS
  \item an MPI implementation
  \item CGNS, a CFD general notation system
  \item libhdf5, a hierarchical data format library used by CGNS
  \item LAPACK, a dense linear algebra package
  \item SUNDIALS-CVODE, a non-linear stiff ODE solver 
\end{itemize}
In a modern GNU/Linux system distribution, say Ubuntu, all the above required tools and packages can
be installed with the command:
\begin{lstlisting}
  # apt-get install cmake gfortran mpi-default-dev mpi-default-bin \
    libcgns-dev libhdf5-dev liblapack-dev libsundials-serial-dev
\end{lstlisting}
Note that the command and the package names may be vary in different GNU/Linux distribution.
If any package is not included the repository of the GNU/Linux distribution on which FOEuler is
deployed, the missing package(s) needs to be manually installed.

Besides the above tools and packages, FOEuler reads unstructured grid generated by GMSH and writes
post-processing data files in the VTK(ParaView) format.
The following command installs these optional external tools:
\begin{lstlisting}
  # apt-get install gmsh paraview
\end{lstlisting}

\subsection{Getting FOEuler}
\label{sec:getting}

The source code of the project FOSolverS is hosted on GitHub.
The home page of the project FOSolverS is \url{https://github.com/mianzhi/fosolvers}.
The full git repository can be downloaded at \\
\url{https://github.com/mianzhi/fosolvers/archive/master.zip}.
In case you are comfortable with git, the version control system used by the project FOSolverS, it
is recommended to clone the full git repository by the following command:
\begin{lstlisting}
  $ git clone https://github.com/mianzhi/fosolvers.git
\end{lstlisting}

\subsection{Building and Installing FOEuler}

The project FOSolverS uses modern build system CMake.
In case that the required packages described in Sec.~\ref{sec:requirement} are properly installed,
and the source are properly downloaded/checked-out as in Sec.~\ref{sec:getting}, the building and
installation of FOEuler is straightforward:
\begin{lstlisting}
  $ cd fosolvers # the top-level CMakeList.txt should in this folder
  $ mkdir build
  $ cd build
  $ cmake ..
  $ make
  # make install
\end{lstlisting}

\section{Using FOEuler}

\subsection{Overview of the Work Flow}

\subsection{Preparing Input Files}

\subsubsection{Grid File}

\subsubsection{Simulation Control File}

\subsubsection{Initial Condition File}

\subsubsection{Boundary Condition File}

\subsubsection{Fluid Property File}

\subsection{Run FOEuler}

Once all the input files are prepared, the FOEuler solver can be started by:
\begin{lstlisting}
  $ cd my_case # this folder contains all input files
  $ foeuler
\end{lstlisting}
The execution of the program may take a long time depending on the size and nature of the problem.
Post-processing files are generated in the same folder as the solver is running.

\end{document}
